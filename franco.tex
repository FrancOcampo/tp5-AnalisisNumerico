\documentclass[a4paper,12pt]{article}

% Packages
\usepackage{amsmath} % For advanced math
\usepackage{amsfonts} % For math fonts
\usepackage{amssymb} % For additional symbols

\title{An Introduction to LaTeX}
\author{Your Name}
\date{\today}

\begin{document}

\maketitle

\section{Introduction}
This is a basic document created using \LaTeX. It includes sections, formulas, and text. 

\subsection{Simple Text}
Here is some plain text. We can also include \textbf{bold text}, \textit{italicized text}, and even \underline{underlined text}.

\section{Mathematics}

\subsection{Inline Formulas}
You can include inline formulas like this: \( E = mc^2 \), which is Einstein's famous energy-mass equivalence.

\subsection{Displayed Equations}
Displayed equations are centered and easier to read:
\[
a^2 + b^2 = c^2
\]

\subsection{Using the Align Environment}
For systems of equations or aligned expressions:
\begin{align*}
x + y &= 10 \\
2x - y &= 5
\end{align*}

\subsection{Summations and Integrals}
LaTeX excels at rendering complex equations:
\[
\sum_{n=1}^\infty \frac{1}{n^2} = \frac{\pi^2}{6}
\]

And integrals:
\[
\int_0^\infty e^{-x^2} \, dx = \frac{\sqrt{\pi}}{2}
\]

\section{Conclusion}
This document showcases the basic structure and some mathematical capabilities of \LaTeX. Try editing this file to include your own text and equations.

\end{document}
